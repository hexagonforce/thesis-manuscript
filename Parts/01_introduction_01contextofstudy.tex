
\section{Context of Study}

Network demands for video-conferencing and other high-bandwidth technology is becoming more complex, as well as the network infrastructure that support these demands. For this reason, it is important to develop and test good Quality of Service (QoS) methods, not only in model, typical networks but also in real-life networks. QoS refers to technology that guarantee high-quality provision of resource-intensive, high-priority services \cite{noauthor_what_nodate}. Requirements of QoS are enforced through queuing, which is the ordering of packets in traffic flow, and bandwidth control, which is how much of the network resources the packets take up \cite{zhao_internet_2000}. In traditional networks with disparate forwarding devices with different configurations, queuing and bandwidth control is enforced through service-level agreements between the user and the network provider \cite{karakus_quality_2017}. This does not allow for abstract and fine-grained control of network traffic.

Software-defined networking (SDN) is an emerging network architecture that solves the problem of fragmented network management and configuration by decoupling the data plane from the control plane \cite{kreutz_software-defined_2015}. Forwarding devices in the network are logically centralized, and uses a protocol such as OpenFlow to control behavior. Software-defined networking allows for QoS provisioning to be implemented with a more abstract view of the network , allowing for better control of different types of traffic. Advantages of the SDN paradigm in providing SDN include custom routing of QoS traffic instead of shortest-path routing, better QoE through additional parameters to the traditional QoS metrics, and better monitoring of network dynamics.

Researchers have tested many QoS algorithms on Mininet, which simulates software-defined networks, with a network operating system such as Ryu and OpenFloodLight \cite{karakus_quality_2017}. Regencia and Yu built a testing framework with Mininet and Ryu to test various QoS algorithms under different network topologies \cite{yang_introducing_2022}. However, the study did not investigate the performance of those QoS algorithms in different topologies. The QoS algorithms can be more thoroughly tested with real-world data from the Internet Topology Zoo (ITZ) \cite{knight_internet_2011}, which provides data from network operators about live networks that provide Internet service. This will allow for simulations that better represent the usage of these algorithms under live networks.