\chapter{RESULTS AND DISCUSSION}
After implementing your methodology and gathering all pertinent data, in this section, you will now present the gathered data to your reader. By the end of this section, your reader should have an idea of what exactlty happened during the experiment. 
A good way to organize your results is to group them is to present them in the same order which your methodology was presented. For instance, if your methodology included the analysis of user logs, the implementation of an application, and the testing of this application, your results should flow in the same way. In addition, more often than not, you will be presenting a large volume of data, so utilize figures and tables whenever appropriate. Table 5.1 below presents one way of how to go about presenting your data. Note the table caption and headers, as mentioned in our framework.

There are, however, some additional notes that must be clarified. First, given that you will be gathering a huge volume of data, you must be able to classify which of these were critical in determining the outcome of your study, and which ones need not be presented. The critical data must be presented in this section, while the minor ones may be placed in the Appendices of your paper, which will be described later in this template.

Another clarification to be noted is that the presentation of results in this section must be objective, or ‘as-is’. This means that you must describe your results in a way understandable to your reader without putting any form of interpretation. In effect, this section’s intent is to provide answers to “what happened” questions, not “what does it mean” questions. The interpretation of results is the subject of a later section.

Finally, because this is a presentation of what happened in the past, all tenses used in this section must be in the past form, be it active or passive. This will also be true for the preceeding sections after the study’s implementation, especially when stating the methodology.