\chapter{RESULTS}
We discuss the convergence times, round-trip times, HTTP stress test benchmark results, and video streaming benchmark results in order, comparing the performances of the different network topologies under the two queuing algorithms tested.

\section{Convergence times}
First, we discuss the convergence time as described in section \ref{3:testing}. Table \ref{tab:convergence} shows the convergence times of the tested topologies. In both the Basic and Source CBQ, in general, larger networks exhibited higher convergence times. The simple class--based queuing algorithm took much longer to converge than the source class-based queuing algorithm in this test.

\begin{table}[htbp]
    \centering
    \begin{tabular}{cccc}
        \toprule
        \multirow{2}{*}{Topology} & \multirow{2}{*}{Network size}& \multicolumn{2}{c}{Convergence Times} \\
        & & Basic CBQ & Source CBQ \\
        \midrule \\

        Agis & 5.370 & 3.950 & \\
        Atmnet & 6.323 & 3.649 & \\
        Canerie & 4.640 & 5.118 & \\
        Gridnet & 3.772 & 3.987 & \\
        Ibm & 3.669 & 4.074 & \\
        Janetbackbone & 3.532 & & \\
        Nsfnet & 3.684 & 4.003 & \\
        Savvis & 3.139 & 3.107 & \\
        Singaren & 3.490 & 3.669 & \\
        WideJpn & 3.848 & 4.074 & \\
        Mesh & 3.047 & 3.671 & \\
        Tree & 3.605 & 3.613 & \\
        \bottomrule
    \end{tabular}
\end{table}

\section{Round-trip times}
Next, we discuss the round trip times after the network has converged. The round trip times are similar in both algorithms, as shown in Tables \ref{tab:cbq_pings} and \ref{tab:sbq_pings}.

\section{VLC traffic performance}

\section{HTTP traffic performance}

