\chapter{CONCLUSION}
\section{Conclusions}
We were able to modify the testing framework to properly accomodate different topologies through the use of the NetworkX library and a simple modification to the Ryu switch algorithm with the Spanning-Tree Protocol.

From the results we provide the following conclusions:
\begin{enumerate}
    \item For round trip times, the two topologies showed similar results; however, the Basic CBQ was able to provide for better service in video traffic, which was the priority traffic in the experiment.
    \item For HTTP file transfer performance, while Basic CBQ was more reliable, with little to no failed requests, raw file transfer rate was higher for Source CBQ in some topologies. 
    \item The performance characteristics across topologies was not purely dependent on the network size. The shape and structure of the topology can also affect the performance metrics.
    \item In Source CBQ, the difference of performance between the different clusters was more pronounced in the two benchmarks, while in Basic CBQ, the trend of increasing performance when moving to smaller-sized clusters was less clear.
    \item Overall, Basic CBQ outperformed Source CBQ in the random sample of ITZ network topologies, as well as the two model topologies selected for testing. 
\end{enumerate}
Regarding point 4, this implies that Basic CBQ was more consistent across all topologies in the video streaming benchmark, which means that Basic CBQ exhibited better performance than Source CBQ.

\section{Recommendations}
The following improvements to this research can be implemented to better understand the behavior of the QoS algorithms in different networks:
\begin{enumerate}
    \item Increase the number of servers and the variety of traffic
    \item Incorporate PCAP into this research to simulate real-life traffic better
    \item Use different path-resolution algorithms other than STP
    \item Increase the scale, both in the size of the network and the number of clients
    \item Investigate algorithms aside from Basic CBQ and Source CBQ, and investigate the differences between leaf-enforced and source-enforced algorithms
    \item Investigate the effect of graph structure and shape, while controlling for the number of nodes.
\end{enumerate}
