\begin{thesisabstract}
\paragraph{         } Quality of Service (QoS) provisioning in Software-defined Networks (SDN) is an active area of research because of the capability of SDNs to control traffic in a centralized, efficient manner independent of the disparate hardware and firmware of the forwarding devices in a network. In this research, the authors extend a previous framework for testing Class-based Queuing (CBQ) algorithms for QoS provisioning with Mininet, Ryu and OpenFlow. We use data from the Internet Topology Zoo which provides real-life network topologies to test the Basic CBQ and Source CBQ algorithms under different network topologies. The ITZ dataset were clustered using a simple K-means clustering algorithm to ensure that a variety of networks are selected. The selected topologies were tested with best-effort HTTP traffic and priority video traffic to measure the performance of the queuing algorithms. The results show that Basic CBQ was able to preserve the performance of priority traffic more effectively than Source CBQ across all topologies, with clusters of larger topology sizes exhibiting a greater performance difference than the clusters of smaller topology sizes. Further comparisons of key performance indicators such as transfer rate and demux bitrate indicate a need for more testing to validate these results and isolate certain variables such as network size, structure, and the choice of path-resolution algorithms.
\end{thesisabstract}
