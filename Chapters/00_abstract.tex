\begin{thesisabstract}
\paragraph{         } Quality of Service (QoS) provisioning in Software-defined Networks (SDN) is an active area of research because of the capability to control traffic in a centralized manner independent of the disparate hardware and firmware of the forwarding devices in a network. In this research, the author extend a previous framework for testing Class-based Queuing (CBQ) algorithms for QoS provisioning with Mininet, Ryu and OpenFlow. We use data from the Internet Topology Zoo which provides real-life network topologies to test the Basic CBQ and Source CBQ algorithms under different network topologies. The ITZ dataset were clustered using a simple K-means clustering algorithm to ensure that a variety of sampled networks. The selected topologies were tested with best-effort HTTP traffic and priority video traffic to measure the performance of the queuing algorithms. Basic CBQ outperformed Source CBQ in best-effort HTTP traffic in terms of reliability across all topologies. In terms of transfer rate however, Basic CBQ performed better than Source CBQ in the smaller Cluster 2 topologies, ($7.74\%-55.6\%$ difference) while Source CBQ had better transfer rates ($24.5\%-27.2\%$ difference) in the larger Cluster 1 and 3 topologies. However, for priority video traffic, Basic CBQ outperformed Source CBQ in all tested topologies, both in terms of demux bitrate and reliability. The results show that Basic CBQ was able to preserve the performance of priority traffic more effectively than Source CBQ across all topologies, with clusters of larger topology sizes exhibiting a greater performance difference between the two algorithms than the clusters of smaller topology sizes. The results demonstrate that the framework is capable of simulating real-life network topologies with expected and consistent results.
\end{thesisabstract}
